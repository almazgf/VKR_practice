\section{реализация проекта}
\subsection{Выбор инструментов разработки}
Исходя из системных требований предъявляемых к реализуемому серверу, к выбору инструментов разработки накладываются следующие ограничения:
\begin{enumerate}
	\item Поддержка формата передачи данных JSON;
	\item Наличие библиотек для работы со стандартом JWT;
	\item Возможность реализации архитектурного стиля REST.
\end{enumerate}
\subsubsection{Выбор инструментов разработки}
Для реализации сервера используется язык программирования Python версии 3.8.5 и фреймворк для создания веб-приложений
Flask. Flask предоставляет гибкую настройку веб-приложения с минимальным количеством зависимостей. Пакетный менеджер pip предоставляет множество расширений для языка Python, благодаря чему реализация взаимодействия между компонентами системы сводится к написанию бизнес логики. Процесс написания программного кода и его отладки осуществлялся в интегрированной среде разработки PyCharm версии 2020.3.3. Для взаимодействия с базой данных используется рекомендованный способ для работы с MongoDB из среды Python пакет PyMongo. Пример подключения к базе данных:
\begin{lstlisting}
app.config["MONGO_URI"] = "mongodb://localhost:27017/barcode"
mongo = PyMongo(app)
\end{lstlisting}
Создается экземпляр класса клиента базы данных, который используется для обращения к данным. Благодаря тому, что данная база является документоориентированной с представлением объектов в типе BSON отсутствует миграция базы данных.

Для реализации стандарта аутентификации используется расширение
Flask-JWT-Extended, которое реализует весь необходимый функционал. Для встраивания данного функционала в программный код, достаточно добавить к методам ресурсов соответствующие декораторы доступа. При возникновении ошибки декоратор прерывает выполнение обработки запроса и возвращает соответствующее сообщение. По умолчанию access-токен валиден в течение 15 минут, а refresh-токен в течение 30 дней. Для авторизованного запроса необходимо передать параметр Authorization в заголовке запроса с содержимым: Bearer <соответствующий токен доступа (access/refresh)>.

\subsubsection{Описание реализованного RestAPI}
Описание основных параметров конфигурации реализованного сервера:
\begin{enumerate}
	\item SECRET KEY -- секретный ключ, который использует Flask при
	вычислении хэша паролей. Пакетное расширение jwt extended использует его для генерации подписи при создании токенов. Секретный ключ хранится отдельно от программного кода и передается посредством переменных окружения (environment variables), при развертке данного приложения на сервере;
	\item JWT ACCESS TOKEN EXPIRES -- срок жизни для токена доступа.  Задаем равной 30 минутам;
	\item JWT REFRESH TOKEN EXPIRES -- срок жизни для токена обновления. Задаем равной 24 часам;
	\item MONGO URI -- URL сервера с MongoDB с необходимыми
	данными авторизации;
\end{enumerate}

Опираясь на модели спроектированные в предыдущей главе, были разработаны методы для каждого ресурса требуемые для полноценной работы сервиса, которым присвоены соответствующие маршруты. В таблице \ref{APIdescription} приводится описание всех реализованных методов. Листинг кода реализованного сервера приведен в приложении А.  

\begin{singlespacing}
\begin{table}[H]
	\caption{Описание реализованного RestAPI}\label{APIdescription}
	\begin{tabular}{|  p{80pt} |  p{80pt} |  p{290pt} |}
	\hline 
	URL & Метод & Описание \\
	\hline
	/registration & POST & Регистрация пользователя с переданными ар-
	гументами. Результатом является сообщение об успешной регистрации пользователя, либо сообщеие о соответствующей ошибке с HTTP-кодом. \\
	\hline
	/login & POST & Аутентификация пользователя по логину и па-
	ролю. Результатом является access- и refresh- токены, либо сообщеие о соответствующей ошибке с HTTP-кодом. \\
	\hline
	/refresh & POST & Обновление access-токена на основе передан-
	ного refresh-токена. Результатом является access-токен. \\
	\hline
	/unacceptable products & POST & Добавление недопустимого продукта для пользователя. Необходим заголовок авторизации. Результатом является сообщение об успешном добавлении продукта, либо сообщеие о соответствующей ошибке с HTTP-кодом. \\
	\hline
	/unacceptable products & DELETE & Удаление недопустимого продукта из списка хранимой в базе данных. Необходим заголовок авторизации. Результатом является сообщение об успешном удалении продукта, либо сообщеие о соответствующей ошибке с HTTP-кодом. \\
	\hline
	/product & POST & Получение информации о продукте по преданному штрих-коду. Необходим заголовок авторизации. Результатом является полнаяинформация о продукте, либо сообщеие о соответствующей ошибке с HTTP-кодом. \\
	\hline
	\end{tabular}
Источник: собственная разработка
\end{table}
\end{singlespacing}
Все ресурсы на стороне сервера требуют авторизации с использованием токенов доступа, за исключением ресурсов, необходимых для аутентификации. Refresh-токен позволяет клиентам запрашивать новые access-токены по истечении их времени жизни. Refresh-токен выдается на более длительный срок, чем access-токен, и по истечению времени его жизни клиенту необходимо вновь пройти процесс аутентификации. 

\clearpage