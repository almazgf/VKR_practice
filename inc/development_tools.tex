\section{Обзор используемых технологий}

\subsection{Архитектура REST}
Термин REST ввел Рой Филдинг, один из создателей протокола HTTP, в своей докторской диссертации "Архитектурные стили и дизайн сетевых программных архитектур" ("Architectural Styles and the Design of Network-based Software Architectures") в 2000 году. REST — это акроним, сокращение от английского Representational State Transfer — передача состояния представления. Архитектурный стиль взаимодействия компонентов распределенной системы в компьютерной сети. REST определяет стиль взаимодействия (обмена данными) между разными компонентами системы, каждая из которых может физически располагаться в разных местах.
Данный архитектурный стиль представляет собой согласованный набор ограничений, учитываемых при проектировании распределенной системы. Эти ограничения иногда называют принципами REST:
\begin{enumerate}
	\item Приведение архитектуры к модели клиент-сервер. В основе данного ограничения лежит разграничение потребностей. Необходимо отделять потребности клиентского интерфейса от потребностей сервера, хранящего данные. Данное ограничение повышает переносимость клиентского кода на другие платформы, а упрощение серверной части улучшает масштабируемость системы. Само разграничение на “клиент” и “сервер” позволяет им развиваться независимо друг от друга; 
	\item Отсутствие состояния. Архитектура REST требует соблюдения следующего условия. В период между запросами серверу не нужно хранить информацию о состоянии клиента. Все запросы от клиента должны быть составлены так, чтобы сервер получил всю необходимую информацию для выполнения запроса и идентификации клиента. Таким образом и сервер, и клиент могут "понимать" любое принятое сообщение, не опираясь при этом на предыдущие сообщения; 
	\item Кэширование. Клиенты могут выполнять кэширование ответов сервера. У тех, в свою очередь, должно быть явное или неявное обозначение как кэшируемых или некэшируемых, чтобы клиенты в ответ на последующие запросы не получали устаревшие или неверные данные. Правильное использование кэширования помогает полностью или частично устранить некоторые клиент-серверные взаимодействия, ещё больше повышая производительность и расширяемость системы;
	\item Единообразие интерфейса. К фундаментальным требованиям REST архитектуры относится и унифицированный, единообразный интерфейс. Клиент должен всегда понимать, в каком формате и на какие адреса ему нужно слать запрос, а сервер, в свою очередь, также должен понимать, в каком формате ему следует отвечать на запросы клиента. Этот единый формат клиент-серверного взаимодействия, который описывает, что, куда, в каком виде и как отсылать и является унифицированным интерфейсом. Каждый ресурс в REST должен быть идентифицирован посредством стабильного идентификатора, который не меняется при изменении состояния ресурса. Идентификатором в REST является URI;
	\item Слои. Под слоями подразумевается иерархическая структура сетей. Иногда клиент может общаться напрямую с сервером, а иногда — просто с промежуточным узлом. Применение промежуточных серверов способно повысить масштабируемость за счёт балансировки нагрузки и распределённого кэширования; 
	\item Код по требованию (необязательное ограничение). Данное ограничение подразумевает, что клиент может расширять свою функциональность, за счет загрузки кода с сервера в виде апплетов или сценариев. 
\end{enumerate}
В общем случае REST является очень простым интерфейсом управления информацией без использования каких-то дополнительных внутренних прослоек. Каждая единица информации однозначно определяется глобальным идентификатором, таким как URL. Каждая URL в свою очередь имеет строго заданный формат. Как происходит управление информацией сервиса – это целиком и полностью основывается на протоколе передачи данных. Для HTTP действие над данными задается с помощью методов: GET (получить), PUT (добавить, заменить), POST (добавить, изменить, удалить), DELETE (удалить).

\subsection{Формат передачи данных JSON}
JSON (JavaScript Object Notation) – это текстовый формат представления данных в нотации объекта JavaScript. Предназначен JSON, также как и некоторые другие форматы такие как XML и YAML, для обмена данными. Несмотря на происхождение от JavaScript, формат считается независимым от языка и может использоваться практически с любым языком программирования. JSON основан на двух структурах данных:
\begin{enumerate}
	\item Коллекция пар ключ/значение;
	\item Упорядоченный список значений.
\end{enumerate}
Это универсальные структуры данных. Почти все современные языки программирования поддерживают их в какой-либо форме. Объект JSON представляет собой заключённый в фигурные скобки список из пар ключ/значение. В коллекции ключ отделяется от значения с помощью знака двоеточия (:), а одна пара от другой - с помощью запятой (,). При этом ключ в JSON обязательно должен быть заключен в двойные кавычки. Значение ключа в JSON можно задать только в одном из следующих форматов: string (строкой), number (числом), object (объектом), array (массивом), boolean (логическим значением true или false) или null. Массив JSON представляет собой заключённый в квадратные скобки список из нуля или более значений, разделённых запятыми. Эти структуры могут быть вложенными.

\subsection{Стандарт JSON Web Token}
JSON Web Token (JWT) — это открытый стандарт (RFC 7519) представления данных для передачи между двумя или более сторонами в виде JSON-объекта. В клиент-серверных приложениях токены создаются сервером, подписываются секретным ключом и передаются клиенту, который в дальнейшем использует данный токен для подтверждения своей личности. Как правило, структурно JWT состоит из трех частей: 
\begin{enumerate}
	\item header — заголовок;
	\item payload — полезная нагрузка;
	\item signature — подпись.
\end{enumerate}
Заголовок и полезная нагрузка — обычные JSON-объекты, которые необходимо дополнительно закодировать при помощи алгоритма base64url. Закодированные части соединяются друг с другом, и на их основе вычисляется подпись, которая также становится частью токена. 
Заголовок — служебная часть токена. Он помогает приложению определить, каким образом следует обрабатывать полученный токен:

\{

	"typ": "JWT",
	
	"alg": "HS256"
	
\}
\newline
Поле typ определяет тип токена. Поле alg определяет алгоритм, использованный для генерации подписи. В приведенном случае был применен алгоритм HS256, в котором для генерации и проверки подписи используется единый секретный ключ. В полезной нагрузке передается любая информация, которая помогает приложению тем или иным образом идентифицировать пользователя и время жизни токена. Заголовок и полезная нагрузка кодируются при помощи алгоритма base64url, после чего объединяются в единую строку с использованием точки (".") в качестве разделителя. Генерируется подпись, которая добавляется к исходной строке так же через точку:
\begin{center}
header.payload.signature
\end{center}
Получив JWT от пользователя, приложение самостоятельно вычислит значение подписи и сравнит его с тем значением, которое было передано в токене. Если эти значения не совпадут, значит, токен был модифицирован или сгенерирован недоверенной стороной, и принимать такой токен и доверять ему приложение не будет. 


\clearpage